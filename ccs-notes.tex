\documentclass{article}

\usepackage{amsthm}
\usepackage{amsmath}
\usepackage{amssymb}
\usepackage{xcolor}
\usepackage{xypic}
\usepackage[left=3cm,right=3cm,bottom=3.2cm,top=3cm]{geometry}
\usepackage{pifont}
\usepackage{mdframed}
\usepackage{url}

\def\mystrut{\rule{0pt}{0.68em}}
\newcommand{\act}[1]{\textsf{#1}}
\newcommand{\da}[1]{\overline{\mystrut\textsf{#1}}}
\newcommand{\defn}{\stackrel{\text{def}}{=}}

\newcommand{\dak}[1]{\overline{\textsf{#1}}}

\newcommand{\op}[2]{\act{r}(#1).\dak{w}(#1 #2)}
\newcommand{\pEnd}{\mathbf{0}}


\newcommand{\highlight}[1]{%
  \colorbox{yellow!50}{$\displaystyle#1$}}
\newcommand{\highlightG}[1]{%
  \colorbox{blue!30}{$\displaystyle#1$}}
\newcommand{\highlightR}[1]{%
  \colorbox{red!25}{$\displaystyle#1$}}
\newcommand{\highlightGH}[1]{%
  \colorbox{blue!5}{$\displaystyle#1$}}

\newcommand{\exampleLabel}{\tag{\small{\textcolor{blue}{\textsf{example}}}}}

\theoremstyle{definition}
\newtheorem{definition}{Definition}
\newtheorem{example}{Example}
\newtheorem{theorem}{Theorem}

\newcommand{\src}[1]{\highlightR{#1}}
\newcommand{\trg}[1]{\highlight{#1}}
\newcommand{\srch}[1]{\highlightG{#1}}
\newcommand{\trgh}[1]{\highlightGH{#1}}
\newcommand{\subd}[1]{\highlight{#1}}

\newcommand{\dual}[1]{\overline{#1}}

\newcommand{\restr}[2]{\nu #2 . #1}

\title{\vspace{-3.5em}CO661 - Theory and Practice of Concurrency \\
{\Large{Notes on the Calculus of Communicating Systems (CCS)}}}

\date{\vspace{-2em}Last updated on \today}

\definecolor{labelColor}{RGB}{201,46,124}
\newcommand{\trule}[1]{\textcolor{labelColor}{\textit{\textsf{#1}}}\;}

\begin{document}
\maketitle{}
\vspace{-1em}

\noindent
In this module, we examine both the theory and practice of
concurrency. The theoretical side brings rigour to our reasoning and
understanding of key concepts in concurrent programming. In this
module, we utilise CCS, the Calculus of Communicating
Systems,\footnote{Robin Milner: \emph{A Calculus of Communicating
    Systems}, Springer Verlag, ISBN 0-387-10235-3. 1980.} as a model
for understanding concurrency. CCS is essentially a small programming
language (a ``\emph{calculus}'') which is carefully and formally
defined.  This allows us to be precise about the meaning of concurrent
programs. CCS is often described as a \emph{process calculus} or
\emph{process algebra} since it is a calculus where ``processes'' are
the central unit of organisation: its programs are processes,
which are made up of further processes.

These notes are open source.\footnote{\url{https://github.com/dorchard/co661-notes.git}} If you spot any
mistakes or errors or things that need clarifying, please fork and
send a pull request.

\tableofcontents

\section{Syntax}

CCS processes comprise \emph{actions}, which we use to abstract
the notion of a \emph{command} in a program or a particular function
call or message. For example, the following CCS program describes a process
that performs an (abstract) action $a$ followed by an (abstract)
action $b$ and then does nothing:
%
\begin{equation*}
  a . b . \mathbf{0}
\end{equation*}
%
Actions sequenced via the operator ``.'' which is called the
\emph{prefixing operator} since it is used to prefix a process
with an action. The process $\mathbf{0}$ is the inactive process.

The full syntax of CCS is defined inductively by the following BNF grammar:
%
\begin{align*}
  \begin{array}{rll}
  P, Q ::= \;\, & \alpha . P & \qquad \textit{prefixing} \\
    \;\,\mid & P + Q & \qquad \textit{non-deterministic choice}  \\
    \;\,\mid &  (P \, | \, Q) & \qquad \textit{parallel composition} \\
    \;\,\mid & \nu a . P & \qquad \textit{name restriction} \\
    \;\,\mid & \mathbf{0} & \qquad \textit{inactive process} \\
    \;\,\mid & P[b/a] & \qquad \textit{relabelling} \\
    \;\,\mid & A & \qquad \textit{process identifiers}
  \end{array}
\end{align*}
%
where  $\alpha ::= a \,\mid\, b \,\mid\, \ldots \overline{a} \,\mid\,
\overline{b} \,\mid\, \ldots$
ranges over \emph{names} (of actions)
where names with an overline are called \emph{co-names} (or \emph{dual
  names}). We use $P$ and $Q$ to range over processes, i.e., when
we want to refer to some process in an abstract sense. We use
$A$ to range over identifiers for processes (i.e., concrete names
given to processes, see below on ``Process definitions'').

The following gives a few examples to give some intuition, but in the
next section we will look at the concrete semantics of proceses
which will make precise what each part of the syntax means.
%
\begin{itemize}
  \item $a. b. P$ -- does action $a$ then action $b$ then continues to
  %
  behave as the process $P$;
  \item $a.0 + b.0$ -- chooses non-deterministically between doing
  action $a$ or doing action $b$.
  %
  \item $(a.b.0 \mid c.d.0)$ -- in parallel, one process does action $a$
  then $b$ then stops (inactive process), whilst the other does $c$
  then $d$ then stops. What is observed is an \emph{interleaving} of
  these actions, e.g., one possibility is that whole process does
  $a$ then $c$ then $b$ then $d$. Another possibility is $c$ then $d$
  then $a$ then $b$. The point is that we may get any interleaving of
  the actions of the subprocesses.
  %
  \item $\nu a . (a . P \mid \overline{a} . Q)$ -- this is a process
  where no $a$ action can be observed externally, because of the name
  restriction $\nu a $. The inner process $(a . P \mid \overline{a}
  . Q)$ can do a \emph{handshake} between the action $a$ and its dual
  $\overline{a}$ which can happen together leaving a process $P \mid Q$.
\end{itemize}
%
\paragraph{Concurrency Workbench}
Note that the Concurrency Workbench tool (CAAL), which we use
in this course, has a slightly different
concrete syntax, with the above constructs represented as follows (in order):
%
\begin{align*}
P ::= \;\ \texttt{a} . P \;\,\mid\;\, \texttt{'a} .P \;\,\mid\;\, P \texttt{+} Q \;\,\mid\;\, (P \texttt{|} Q)
\;\,\mid\;\, P \, \texttt{$\backslash\{$}\texttt{a1}, \ldots, \texttt{an}\texttt{$\}$}
\;\,\mid\;\, \texttt{0} \;\,\mid\;\, P\texttt{[b/a]} \;\,\mid\;\, A
\end{align*}

\paragraph{Process definitions}

We will defined processes by recursive definitions. For example, the
following defines a process named $A$:
%
\begin{equation*}
A \; {\mathrel{\mathop{=}\limits^{{def}}}} \; a.b.A
\end{equation*}
%
The process is recursively defined, and performs actions $a$ then $b$ 
continuously.

Processes can be mutually recursive. For example, we could
define a process that performs $a$ then $b$ actions
continuously via two definitions:
%
\begin{equation*}
  A \; {\mathrel{\mathop{=}\limits^{{def}}}} \; a.B
  \qquad\qquad
  B \; {\mathrel{\mathop{=}\limits^{{def}}}} \; b.A
\end{equation*}
%
Here $A$ performs $a$ then $b$ actions
continuously whilst $B$ performs $b$ then $a$ actions
continuously (they have different starting points).

\vspace{-1em}
\section{Semantics}

We describe the semantics (the meaning) of CCS processes via a
\emph{labelled-transitions system} which effectively describes small
``steps'' of evaluation for CCS processes along with actions which are
made observable by this evaluation.  We define these small steps
(called \emph{reductions} or \emph{transitions})
by a relation between processes,\footnote{Viewed set theoretically,
  $\rightarrow$ is a ternary relation, i.e.
  $(\rightarrow) \subset P \times l \times P$.}  annotated with a
label, of the form:
%
\begin{equation*}
P \xrightarrow{l} Q
\end{equation*}
%
which describes that a process $P$ can reduce (\emph{evaluate}) to a
process $Q$ by doing some action described by the label $l$.
This is a \emph{labelled transition}.
Labels are as defined as:
$$l ::= \alpha \,\mid\, \tau$$
i.e., they are
either names of actions or they are a special label $\tau$ called a
\emph{silent action}.

Since the syntax of processes is itself inductively defined (i.e.
processes can contain processes) so the definition of $\rightarrow$ is
inductively defined. We leverage the \emph{inference rule} style, with
inductive rules having one or more premises and axiom rules with no
premises. Each piece of syntax has at least one corresponding rule for
$\rightarrow$, giving the semantics of each syntactic construct. 

\paragraph{Actions}

The first simple reduction rule is for action prefixes:
%
\begin{align*}
  \trule{action}\dfrac{}{\alpha.P \xrightarrow{\alpha} P}
\end{align*}
%
We see that a process with an action $a$ prefixing another process $P$ can reduce by
emitting that action and then becoming the remaining processes $P$.
e.g.
%
\begin{align*}
  \exampleLabel{}
  \trule{action}\dfrac{}{a.b.\mathbf{0} \xrightarrow{a} b.\mathbf{0}}
\end{align*}
%
\begin{definition}[Trace]
  Labelled transitions can be composed together to form a
  \emph{trace},
  which comprises a sequence of reductions steps, e.g.,
  %
  \begin{equation*}
    a.b.\mathbf{0} \; \xrightarrow{a} \; b.\mathbf{0} \; \xrightarrow{b} \;\mathbf{0}
  \end{equation*}
  %
  Each step here is a reduction that needs a \emph{derivation}. We
  will see below that we can derive reductions for more complex
  processes by plugging together inference rules.
\end{definition}

\paragraph{Parallel composition}
A parallel composition of processes $P \mid Q$ has two possible
reductions given by the following inductive rules:
%
\begin{align*}
  \trule{par1}{\dfrac{P \xrightarrow{l} P'}{P \mid Q \xrightarrow{l} P' \mid Q}}
\qquad
\trule{par2}{\dfrac{Q \xrightarrow{l} Q'}{P \mid Q \xrightarrow{l} P \mid Q'}}
\end{align*}
%
The \trule{par1} rule says that if we have a process $P$ that
can reduce to $P'$ with label $l$ then we can place $P$ in parallel
with $Q$ (i.e., $P \mid Q$) and reduce to $P' \mid Q$ with label $l$.
Essentially we are allowing a process within a parallel composition
to make some progress. The \trule{par2} provides the symmetric case.

\begin{example}
We can compose the \trule{action} and \trule{par1}
rules to get the following derivation of a single reduction
step for the process $a . P \mid b . Q$:
%
\begin{align*}
  \exampleLabel{}
  \trule{par1}\dfrac{\trule{action}
  \dfrac{}{a.P \xrightarrow{a} P}}{a.P \mid b.Q \xrightarrow{a} P \mid b.Q}
\end{align*}
%
Thus, we have reduced on the left-hand side of the parallel
composition, emitting an $a$ action to get the resulting process $P
\mid b . Q$. This process could then reduce by applying \trule{par2}:
%
\begin{align*}
  \exampleLabel{}
  \trule{par2}\dfrac{\trule{action}
   \dfrac{}{b.Q \xrightarrow{b} Q}}{P \mid b.Q \xrightarrow{b} P \mid Q}
\end{align*}
%
These two reduction steps, once derived, can be put together to form
the following trace:
%
\begin{equation*}
a.P \mid b.Q \xrightarrow{a} P \mid b.Q \xrightarrow{b} P \mid Q
\end{equation*}
%
This is not the only possible reduction sequence for
the process $(a.P \mid b.Q)$. Another possibility is that we could
reduce on the right first, then on the left, giving the trace:
%
\begin{equation*}
a.P \mid b.Q \xrightarrow{b} a.P \mid Q \xrightarrow{a} P \mid Q
\end{equation*}
%
We can put these two traces together into a \emph{transition graph} that shows
the possible traces:
%
\begin{equation*}
  \xymatrix{
    a.P \mid b.Q \ar[r]^{a} \ar[d]_-{b} & P \mid b.Q \ar[d]^-{a} \\
    a.P \mid Q \ar[r]_{a} & P \mid Q
    }
\end{equation*}
%
Note that this represents a \emph{confluent} reduction, i.e., we take
different paths in the reduction but end in the same place.
\end{example}

A further reduction is possible for a parallel composition of processes,
but in a restricted setting where two parallel processes reduce
by dual actions:
%
\begin{align*}
  \trule{handshake}\dfrac{P \xrightarrow{a} P' \qquad Q \xrightarrow{\overline{a}} Q'}
  {P \mid Q \xrightarrow{\tau} P' \mid Q'}
\end{align*}
%
In this case we have the processes $P$ and $Q$ making a reduction step
\emph{at the same time} to become $P' \mid Q'$ if they reduce by dual
actions $a$ and $\overline{a}$. We can use this to model notions of
synchronisation, co-ordination, or communication between processes.

\begin{example}
  Consider the following processes representing a vending machine
  $V$ and a customer $C$:
  %
  \begin{align*}
V  & {\mathrel{\mathop{=}\limits^{{def}}}} 
     {\mathsf{coin}}.\overline{\mathsf{tea}}.V \\
C & {\mathrel{\mathop{=}\limits^{{def}}}} 
\overline{\mathsf{coin}}.{\mathsf{tea}}.C   
  \end{align*}
  %
  Since $V$ can reduce by emitting the action $\mathsf{coin}$ and
  $C$ can reduce by emitting the action $\overline{\mathsf{coin}}$
   we can get the reduction:
  %
  \begin{align*}
    \exampleLabel{}
    \trule{handshake}
    \dfrac{V \xrightarrow{\mathsf{coin}} \overline{\mathsf{tea}}.V
    \qquad
    C \xrightarrow{\overline{\mathsf{coin}}} \mathsf{tea}.C}
    {V \mid C \xrightarrow{\tau} (\overline{\mathsf{tea}}.V \mid \mathsf{tea}.C)}
  \end{align*}
  %
  Note that the resulting process can further do a handshake because
  of the dual $\mathsf{tea}$ actions, bringing us back to the process
  $V \mid C$.
\end{example}

\paragraph{Choice}
A non-deterministic choice between two processes $P + Q$ reduces
by reducing just one side, resulting in a process that corresponds
to just one side of the choice, given by the rules:
%
\begin{align*}
\trule{choice1}\dfrac{P_1 \xrightarrow{l} Q}{P_1 + P_2 \xrightarrow{l} Q}
\qquad
\trule{choice2}\dfrac{P_2 \xrightarrow{l} Q}{P_1 + P_2 \xrightarrow{l} Q} 
\end{align*}
%
\begin{example} For a process $b.P + c.Q$, the top-level syntactic
  construct is non-deterministic choice, so a reduction must necessarily
   use \trule{choice1} or \trule{choice2}. For example:
  \begin{align*}
    \exampleLabel{}
\trule{choice2}
  \dfrac{\trule{action}\dfrac{}{c . Q \; \xrightarrow{c} \; Q}}
        {b . P + c.Q \; \xrightarrow{c} \; Q}
\end{align*}
\end{example}
%
\noindent
Note that when we make a choice, we lose the other branch--- above,
$b.P$ disappears in the resulting process term. The lectures have more examples.

\paragraph{Restriction}

The notion of restricting a name for a process, i.e. $\nu a . P$
can be thought of a bit like binding a name $a$ in the scope of $P$
so that the name is ``local'' to $P$ and cannot be observed outside of
it. Its semantics is given by:
%
\begin{align*}
  \trule{restriction}\dfrac{P \xrightarrow{l} P'}
{\nu a . P \xrightarrow{l} \nu a . P'} {l \neq a \wedge l \neq \overline{a}}
\end{align*}
%
Here we are saying that we can observe the action given
by label $l$ only if it is not the bound name $a$ or its dual
$\overline{a}$. We can use this to enforce handshaking
of dual actions.

\begin{example}
  Consider the processes $a . P$ and $\overline{a} . Q$.
  Their parallel composition has three possible traces:
  %
  \begin{align*}
    & (a . P \mid \overline{a} . Q) \xrightarrow{a}
    (P \mid \overline{a} . Q) \xrightarrow{\overline{a}}
    (P \mid Q) \\
    & (a . P \mid \overline{a} . Q) \xrightarrow{\overline{a}}
    (a.P \mid Q) \xrightarrow{a}
      (P \mid Q) \\
    & (a . P \mid \overline{a} . Q) \xrightarrow{\tau}
      (P \mid Q)
   \end{align*}
   %
   By restricting the process on action $a$ (i.e., the process term
   $\nu a . (a . P \mid \overline{a} . Q)$) we can make sure that the
   handshaking trace is the only one possible, preventing the other two traces, with
   the derivation:
   %
   \begin{align*}
          \exampleLabel{}
     \trule{restriction}\dfrac{
     \trule{handshake}
     \dfrac{
       \trule{action}\dfrac{}{a . P \xrightarrow{a} P}
       \quad
       \trule{action}\dfrac{}{\overline{a} . Q \xrightarrow{\overline{a}} Q}
     }{(a . P \mid \overline{a} . Q) \xrightarrow{\tau} P \mid Q}
     }{\nu a . (a . P \mid \overline{a} . Q) \xrightarrow{\tau} \nu a
     . (P \mid Q)}
   \end{align*}
   %
   The side condition for \trule{restriction} is satisfied since
   $a \neq \tau$ and $\dual{a} \neq \tau$. No other derivation is
   possible. For example, the following is not a valid derivation
   because it would violate the side condition of \trule{restriction}
   (that the label does not match the name being restricted over):
   %
   \begin{align*}
     \exampleLabel{}
     {\Large{{\textcolor{red}{\text{\ding{55}}}}}} \qquad
     \trule{restriction}\dfrac{
     \trule{par}
     \dfrac{
       \trule{action}\dfrac{}{a . P \xrightarrow{a} P}
      }{(a . P \mid \overline{a} . Q) \xrightarrow{a} (P \mid
  \overline{a} . Q)}
     }{\nu a . (a . P \mid \overline{a} . Q) \xrightarrow{a} \nu a
     . (P \mid \dual{a} . Q)}
   \end{align*}
   %
\end{example}
%
We can encapsulate the above notion via a theorem:
%
\begin{theorem}\label{thm:forcing-handshaking}
  Restriction of a name forces handshaking on dual names.
  That is:
  %
    \begin{align*}
      \forall P, P', Q, Q', a . \qquad &
    P \xrightarrow{a} P' \wedge
   Q \xrightarrow{\dual{a}} Q' \; \wedge
      P \neq P' \wedge Q \neq Q' \\
    \Leftrightarrow \;\;\; &
\nu a . P \mid Q \, \xrightarrow{\tau} \, \nu a . P' \mid Q'
    \end{align*}
  %
    That is, if two process can reduce by emitting dual names $a$ and
    $\dual{a}$, then their parallel composition under a restriction of
    $a$ can only reduce by a handshake, and vice versa.\footnote{
      The lecture stated this slightly differently with:
  \begin{align*}
    \forall P, P', Q, Q', a . \qquad &
    P \mid Q \; \xrightarrow{a} \; P' \mid Q \;
    \xrightarrow{\overline{a}} \; P' \mid Q' \\[-0.4em]
   \vee \;\;\;  & P  \mid Q \; \xrightarrow{\overline{a}} \; P \mid Q'
      \; \xrightarrow{{a}} \; P' \mid Q' \\
    \Leftrightarrow \;\;\; &
\nu a . P \mid Q \, \xrightarrow{\tau} \, \nu a . P' \mid Q'
  \end{align*}
This is essentially the same as the theorem here, but more complex
than necessary. Note that we also make it clear here that
$P$ and $P'$ must be distinct in order to get the bi-implication.}
\end{theorem}
%
Finally, a common syntactic shorthand for a several name restrictions is to
give a comma-separated list of the restricted names, e.g.
$\nu a . \nu b . P$ is written as $\nu a, b . P$.

\paragraph{Process identifiers}

A process identifier $A$ can reduce given a definition for
$A$ which itself can reduce. This is given by the rule:
%
\begin{align*}
\trule{def}
\dfrac{P \xrightarrow{l} Q}
      {A \xrightarrow{l} Q} \;\; (A {\mathrel{\mathop{=}\limits^{\textit{def}}}} P)
\end{align*}
%
\begin{example}
For example, if we have $\mathit{A} \defn{} a.b.A$ then we get
the derivation:
\begin{align*}
  \exampleLabel{}
  \trule{def}\dfrac{\trule{action}\dfrac{}{a.b.A \xrightarrow{a} b.A}}
                  {A \xrightarrow{a} b.A}
\end{align*}
\end{example}

\paragraph{Relabelling}

Relabelling give us a way to rename an emitted action.
That is $P [b/a]$ means that for process $P$, any emitted action $a$
is renamed to the action $b$. The semantics is given by
the reduction:
%
\begin{align*}
  \trule{relabel}
\dfrac{P \xrightarrow{l} Q}
  {P[b/a] \xrightarrow{l[b/a]} P[b/a]}
\end{align*}
%
where relabelling $l[b/a]$ means to relabel name $a$ to $b$ for the label $l$, defined:
%
\begin{align*}
l[b/a] = \begin{cases}
b & \textit{when} \;l = a \\
\overline{b} & \textit{when} \; l = \overline{a} \\
l & \textit{otherwise}
\end{cases}
\end{align*}
%
Relabelling is useful when reusing definitions for different purposes
(abstraction and reuse), but we do not make frequent use of it in the course.

\subsection{Collected rules}

\begin{mdframed}
\begin{center}
\begin{align*}
\begin{array}{c}
\trule{action}\dfrac{}{\alpha.P \xrightarrow{\alpha} P} \\[2em]
%
\trule{par1}{\dfrac{P \xrightarrow{l} P'}{P \mid Q \xrightarrow{l} P' \mid Q}}
\qquad
\trule{par2}{\dfrac{Q \xrightarrow{l} Q'}{P \mid Q \xrightarrow{l} P \mid Q'}} \\[2em]
\trule{handshake}\dfrac{P \xrightarrow{a} P' \qquad Q \xrightarrow{\overline{a}} Q'}
{P \mid Q \xrightarrow{\tau} P' \mid Q'}
\qquad
\trule{restriction}\dfrac{P \xrightarrow{l} P'}
{\nu a . P \xrightarrow{l} \nu a . P'} {l \neq a \wedge l \neq \overline{a}} \\[2em]
\trule{choice1}\dfrac{P_1 \xrightarrow{l} Q}{P_1 + P_2 \xrightarrow{l} Q}
\qquad
\trule{choice2}\dfrac{P_2 \xrightarrow{l} Q}{P_1 + P_2 \xrightarrow{l} Q} \\[2em]
\trule{def}
\dfrac{P \xrightarrow{l} Q}
      {A \xrightarrow{l} Q} \;\; (A {\mathrel{\mathop{=}\limits^{\textit{def}}}} P)
\qquad
\trule{relabel}
\dfrac{P \xrightarrow{l} Q}
      {P[b/a] \xrightarrow{l[b/a]} P[b/a]}
\end{array} \\
\end{align*}
where relabelling $l[b/a]$ means to relabel name $a$ to $b$ for the label $l$, defined:
%
\begin{align*}
l[b/a] = \begin{cases}
b & \textit{when} \;l = a \\
\overline{b} & \textit{when} \; l = \overline{a} \\
l & \textit{otherwise}
\end{cases}
\end{align*}
\end{center}
\end{mdframed}

\begin{example}
  Here is an extended example of a 
  derivation of one possible reduction for the process
defined $A \stackrel{\textit{def}}{=} a . \mathbf{0} \mid (b . P + c.Q)$:
%
\begin{align*}
  \exampleLabel{}
  \trule{def}
  \dfrac{
\trule{par2}
\dfrac{
\trule{choice2}
\dfrac{\trule{action}\dfrac{}{c . Q \; \xrightarrow{c} \; Q}}
{b . P + c.Q \; \xrightarrow{c} \; Q}}
{a . \mathbf{0} \mid (b . P + c.Q) \; \xrightarrow{c} \; a.\mathbf{0}
  \mid Q}
  }
  {A \; \xrightarrow{c} \; a.\mathbf{0} \mid Q}
\end{align*}
\end{example}

\section{Modelling locks, semaphores, and race conditions}

We can use CCS to model common concurrent programming constructs,
helping us to understand their behaviour. The following accompanies
the lectures for this concept.

The following is a simple CCS model for the
vending machine of \texttt{TeaRoomInitial.java}:
%
\setlength{\arraycolsep}{0.2em}
\begin{align*}
\begin{array}{rrcl}
%%
(\textit{accounting})\quad &
\textit{Inc}(p).P & \defn{} & \op{\textit{p}}{+1}.P \\
& \textit{Dec}(s).P & \defn{} & \op{\textit{s}}{-1}.P \\
& A.P & \defn{} & \textit{Dec}(s).\textit{Inc}(p).P \\[0.75em]
%%
%%
(\textit{vend})\quad & V & \defn{} & \act{coin}.\da{lock}.A.\da{unlock}.\da{tea} \\
%%
(\textit{machine})\quad & M & \defn{} & V \mid M
%%
\end{array}
\end{align*}
%
The ``{accounting}'' agent $A$ provides the behaviour of reading and
then updating two variables: incrementing the profit $p$ and
decrementing the supply $s$. Note that we abuse notation slightly here
and definition \emph{parametric} processes $\mathit{Inc}(p)$ and
$\mathit{Dec}(s)$ parameterised by the name of some variable we are
changing.  This is not real CCS syntax, but a bit of meta-level
syntactic sugar to simplify the model.

The vending
machine receives a $\act{coin}$, then requests the lock before trying to behave as an accounting
agent, then unlocking and sending tea.
Since a machine is a shared object for which the
\texttt{vend} method can be called arbitrarily and concurrently,
a machine $M$ is an infinite parallel composition of $V$.



A customer and a lock (a shared object) are defined:
%
\setlength{\arraycolsep}{0.2em}
\begin{align*}
\begin{array}{rrcl}
(\textit{lock})\qquad & L & \defn{} & \act{lock}.\act{unlock}.L \\
(\textit{customer})\qquad & C & \defn{} & \da{coin}.\act{tea}.C
\end{array}
\end{align*}

\noindent
The following CCS agent then models two consumers at the vending machine:
%
\begin{align*}
  & \nu \act{coin} . \; \nu \act{tea} . \; \nu \act{lock} . \;
\nu \act{unloack} . \; {({C} \mid C \mid {M} \mid L)}
\end{align*}
  %
We are \emph{restricting} the names \act{coin}, \act{tea},
\act{lock} and \act{unlock} so that handshaking is guaranteed
(see Theorem~\ref{thm:forcing-handshaking}).
The following gives a trace of the model, but we elide the restriction
since it is not necessary to get the following trace.

\renewcommand{\restr}[2]{#1}

\iffalse
\newcommand{\asrch}[1]{#1}
\newcommand{\atrgh}[1]{#1}
\newcommand{\asrc}[1]{#1}

\begin{align*}
 \restr{\asrch{C} \mid C \mid \asrch{M}}{\actions{}} \;\;
\equiv \;\;
 \restr{\asrch{C} \mid C \mid V \mid V \mid \asrch{M}}{\actions{}} \;\;
\equiv \;\; & \restr{\atrgh{\asrc{\da{coin}}.\act{tea}.C} \mid \asrc{\da{coin}}.\act{tea}.C \mid \atrgh{\asrc{\act{coin}}.A.\da{tea} \mid \asrc{\act{coin}}.A.\da{tea} \mid M}}{\actions{}} \\
%%%%
\xrightarrow{\tau} \;\; & \restr{\act{tea}.C \mid \asrc{\da{coin}}.\act{tea}.C \mid A.\da{tea} \mid \asrc{\act{coin}}.A.\da{tea} \mid \asrch{M}}{\actions{}}  \\
%%%%%
\xrightarrow{\tau} \;\; &
 \restr{\act{tea}.C \mid {\act{tea}.C} \mid
A.\da{tea} \mid
A.\da{tea} \mid M}{\actions{}}
\end{align*}
\fi

A process $P$ is highlighted as $\srch{P}$ when we are going to expand
its definition in the next line, where $\trgh{P}$ marks the expanded
definition. Pairs of dual (complementary) actions are highlighted as $\src{\act{a}}$
and $\src{\dak{a}}$ when they are about to handshake via a
$\xrightarrow{\tau}$ step.
%%

\newcommand{\actions}{(\ldots)}
%
\begin{align*}
 \restr{\srch{C} \mid C \mid \srch{M} \mid L}{\actions{}} \;\,
\equiv \;\; & \restr{\trgh{\src{\da{coin}}.\act{tea}.C} \mid C \mid \trgh{\src{\act{coin}}.\da{lock}.A.\da{unlock}.\da{tea} \mid M} \mid L}{\actions{}} \\
%%%%%
\xrightarrow{\tau} \;\; & \restr{{\act{tea}.C} \mid C \mid {{\da{lock}}.A.\da{unlock}.\da{tea} \mid M} \mid \srch{L}}{\actions{}} \\
%%%%
\equiv \;\; & \restr{{\act{tea}.C} \mid C \mid {{\src{\da{lock}}}.A.\da{unlock}.\da{tea} \mid M} \mid \trgh{\src{\act{lock}}.\act{unlock}.L}}{\actions{}} \\
%%%%%
\xrightarrow{\tau} \;\; & \restr{\act{tea}.C \mid \srch{C} \mid A.\da{unlock}.\da{tea} \mid \srch{M} \mid \act{unlock}.L}{\actions{}}  \\
%%%%%
\equiv \;\; & \restr{\act{tea}.C \mid \trgh{\src{\da{coin}}.\act{tea}.C} \mid A.\da{unlock}.\da{tea} \mid \trgh{\src{\act{coin}}.\da{lock}.A.\da{unlock}.\da{tea} \mid M} \mid \act{unlock}.L}{\actions{}} \\
%%%%%
\xrightarrow{\tau} \;\; & \restr{\act{tea}.C \mid {\act{tea}.C} \mid
                          \underbrace{A.\da{unlock}.\da{tea}}_{\text{vend
                          1}} \mid
                          \underbrace{\subd{\da{lock}}.A.\da{unlock}.\da{tea}}_{\text{vend 2}}\mid M \mid \act{unlock}.L}{\actions{}}
\end{align*}
%
The second vending process is not
able to acquire the lock since there is no dual $\act{lock}$ action
that is the head prefix of any other process. The ``vend 2'' process has to wait for
the first one (``vend 1'') to unlock via $\da{unlock}$ which then puts the lock $L$
back to its initial configuration. That is, after some steps dealing
with accounting (which is now atomic) we get:
%
\begin{align*}
   & \restr{\act{tea}.C \mid {\act{tea}.C} \mid
                          \underbrace{\da{tea}}_{\text{vend
                          1}} \mid
                          \underbrace{{\da{lock}}.A.\da{unlock}.\da{tea}}_{\text{vend
                          2}}\mid M \mid
  \srch{L}}{\actions{}} \\
 \equiv \;\; & \restr{\act{tea}.C \mid {\act{tea}.C} \mid
                          {\da{tea}} \mid
                          {\src{\da{lock}}.A.\da{unlock}.\da{tea}} \mid M \mid
                \trgh{\src{\act{lock}}.\act{unlock}.L}}{\actions{}}
\end{align*}
%
\noindent
We define the following CCS model of \texttt{TeaRoomSem.java} which reuses
the accounting agent $A$:
%
\setlength{\arraycolsep}{0.2em}
\begin{align*}
\begin{array}{rrcl}
%%
%(\textit{accounting})\qquad &
%\textit{Inc}(p).P & \defn{} & \op{\textit{p}}{+1}.P \\
%& \textit{Dec}(s).P & \defn{} & \op{\textit{s}}{-1}.P \\
%& A.P & \defn{} & \textit{Dec}(s).\textit{Inc}(p).P \\
%%%
(\textit{1-semaphore})\qquad & S_1 & \defn{} & \act{acq}.\act{rel}.S_1 \\
%%
(\textit{n-semaphore})\qquad & S_n & \defn{} & S_1 \mid S_{n-1} \\[0.5em]
%%
(\textit{vend})\qquad & V & \defn{} & \act{coin}.\da{acq}.A.\da{rel}.\da{tea} \\
%%
(\textit{machine})\qquad & M & \defn{} & V \mid M \\
%%
(\textit{customer})\qquad & C & \defn{} & \da{coin}.\act{tea}.C \\
\end{array}
\end{align*}

\noindent
A semaphore with $n$ permissions (capacity) is modelled by $S_n$ which
composes in parallel $n$ copies of the binary semaphore agent
$S_1$. We can see that a binary semaphore is basically the same as our
previous model of a lock $L$.  However, now $n$ processes can be
in their critical section.

The following trace models two consumers at a vending machine
with a semaphore that has two permissions:
%
%\renewcommand{\srch}[1]{#1}
%\renewcommand{\trgh}[1]{#1}
%\renewcommand{\subd}[1]{#1}
%\renewcommand{\src}[1]{#1}

\begin{align*}
  \restr{\srch{C} \mid C \mid \srch{M} \mid S_2}{(\act{coin}, \act{tea}, \da{acq}, \act{rel}} \;\,
\equiv \;\; & \restr{\trgh{\src{\da{coin}}.\act{tea}.C} \mid C \mid \trgh{\src{\act{coin}}.\da{acq}.A.\da{rel}.\da{tea} \mid M} \mid S_2}{(\ldots)} \\
%%%%%
\xrightarrow{\tau} \;\; & \restr{{\act{tea}.C} \mid C \mid {{\da{acq}}.A.\da{rel}.\da{tea} \mid M} \mid \srch{S_2}}{(\ldots)} \\
%%%%
\equiv \;\; & \restr{{\act{tea}.C} \mid C \mid {{\src{\da{acq}}}.A.\da{rel}.\da{tea} \mid M} \mid \trgh{\src{\act{acq}}.\act{rel}.S_1 \mid S_1}}{(\ldots)} \\
%%%%%
\xrightarrow{\tau} \;\; & \restr{\act{tea}.C \mid \srch{C} \mid A.\da{rel}.\da{tea} \mid \srch{M} \mid \act{rel}.S_1
\mid {S_1}}{(\ldots)}  \\
%%%%%
\equiv \;\; & \restr{\act{tea}.C \mid \trgh{\src{\da{coin}}.\act{tea}.C} \mid A.\da{rel}.\da{tea} \mid \trgh{\src{\act{coin}}.\da{acq}.A.\da{rel}.\da{tea} \mid M} \mid \act{rel}.S_1
\mid S_1}{(\ldots)} \\
%%%%%
\xrightarrow{\tau} \;\; & \restr{\act{tea}.C \mid {\act{tea}.C} \mid A.\da{rel}.\da{tea} \mid {\da{acq}.A.\da{rel}.\da{tea} \mid M} \mid \act{rel}.S_1
\mid \srch{S_1}}{(\ldots)} \\
%%%%%
\equiv \;\; & \restr{\act{tea}.C \mid {\act{tea}.C} \mid A.\da{rel}.\da{tea} \mid {\src{\da{acq}}.A.\da{rel}.\da{tea} \mid M} \mid \act{rel}.S_1
\mid \trgh{\src{\act{acq}}.\act{rel}.S_1}}{(\ldots)} \\
%%%%%%
\xrightarrow{\tau} \;\; & \restr{\act{tea}.C \mid {\act{tea}.C} \mid \subd{A.\da{rel}.\da{tea} \mid A.\da{rel}.\da{tea}} \mid M \mid \act{rel}.S_1
\mid {\act{rel}.S_1}}{(\ldots)}
\end{align*}
%
We now have a race condition in the reduction of the yellow highlighted
sub-process, which we show separate from the rest of the context
(which is independent) setting $Q = \da{rel}.\da{tea}$:
%
\begin{align*}
\\[-1.5em]
\begin{array}{c}
\xymatrix@C=-2em@R=2em{
& & A.Q \mid A.Q \ar[dl]_{\act{r}(s)} \ar[dr]^{\act{r}(s)} & & \\
& \da{w}(s-1).\textit{Inc}(p).Q \mid A.Q
\ar@{->}[dl]_(.65){\dak{w}(s-1)} \ar@{->}[dr]^(.5){\act{r}(s)}
 & & A.Q \mid \da{w}(s-1).\textit{Inc}(p).Q \ar@{->}[dl]_(.5){\act{r}(s)} \ar@{->}[dr]^(.65){\dak{w}(s-1)} & \\
\textit{Inc}(p).Q \mid A.Q & & \subd{\da{w}(s-1).\textit{Inc}(p).Q \mid \da{w}(s-1).\textit{Inc}(p).Q} &
 & A.Q \mid \textit{Inc}(p).Q
}
\end{array}
\\[-1.5em]
\end{align*}

\noindent
Note that the middle term is achieved when the left and right
hand processes interleave their reads, leading to a state where
both processes have read the same value of $s$ (the supply) and will write the
same value of $s-1$, shadowing the fact that \textbf{two} processes are actually
reducing the supply.

Furthermore, this can hide the situation where the supply is only one
($s = 1$) and therefore only one process is allowed to consume from the supply.
But if both processes interleave their reads as in the above, they will both
see a view of the world where $s = 1$ and erroneously assume there is
enough in the supply for them both to proceed.

Thus we see that semaphores alone cannot prevent \emph{race conditions}.
If semaphore-guarded code contains shared state, we must also enforce
mutual exclusion on that shared state via an additional binary semaphore
or mutual-exclusion lock.

\iffalse

\begin{equation*}
{\mathsf{coin}}.\mathsf{acq}.A.\overline{\mathsf{rel}}.\overline{\mathsf{tea}}
\mid
{\overline{\mathsf{coin}}}.\mathsf{tea}.\mathsf{wait}.C
\mid
S_1
\end{equation*}

%\iffalse
\begin{align*}
C \mid V \mid S_1
%%%%
\; \equiv \;
%%%%
\src{\mathsf{coin}}.\mathsf{acq}.A.\overline{\mathsf{rel}}.\overline{\mathsf{tea}}
\mid
\src{\overline{\mathsf{coin}}}.\mathsf{tea}.\mathsf{wait}.C
\mid
S_1
%%%%
& \,\; \xrightarrow{\tau} \;\,
%%%
\mathsf{acq}.A.\overline{\mathsf{rel}}.\overline{\mathsf{tea}}
\mid \mathsf{tea}.\mathsf{wait}.C
\mid \srch{S_1} \\
%%%%
& \,\; \equiv \;\,
\src{\mathsf{acq}}.A.\overline{\mathsf{rel}}.\overline{\mathsf{tea}}
\mid \mathsf{tea}.\mathsf{wait}.C
\mid \src{\act{acq}}.\act{rel}.S_1 \\
%%%
& \,\; \xrightarrow{\tau} \;\,
\src{A}.\overline{\mathsf{rel}}.\overline{\mathsf{tea}}
\mid \mathsf{tea}.\mathsf{wait}.C
\mid \act{rel}.S_1 \\
%%%
& \,\; \rightarrow^\ast \;\,
\src{\overline{\mathsf{rel}}}.\overline{\mathsf{tea}}
\mid \mathsf{tea}.\mathsf{wait}.C
\mid \src{\act{rel}}.S_1 \\
& \,\; \rightarrow^\tau \;\,
\overline{\mathsf{tea}}
\mid \mathsf{tea}.\mathsf{wait}.C
\mid S_1 \\
\end{align*}
%\fi

\fi

\section{Structural congruence}

A basic notion of equality can be ascribed to processes which explains
when processes are treated as being the same. This equality
relation, which we write as $\equiv$, is known
as  \emph{structural congruence}. Which we define in chunks
here based on the main syntactic construct of interest.

Firstly, for parallel composition, structural congruence is defined:
%
\begin{align*}
  P \mid Q & \equiv Q \mid P \\
    (P \mid Q) \mid R & \equiv P \mid (Q \mid R) \\
    P \mid \mathbf{0} & \equiv P \\
\end{align*}
i.e., parallel composition is commutative and associative,
and has the inactive process $\mathbf{0}$ as a unit;
parallel composition is a commutative monoid with $\mathbf{0}$.

For choice, we get the equations:
%
\begin{align*}
 P + Q & \equiv Q + P \\
(P + Q) + R & \equiv P + (Q + R) \\
 P + \mathbf{0} & \equiv P \\
 P + P & \equiv P \\
\end{align*}
Thus, choice is commutative, associative, has
$\mathbf{0}$ as a unit, and is also idempotent:
choosing between the same thing is no choice at all.

Finally, for restriction we have some more involved rules
with side conditions, that related to how we can distribute
bindings (that is, move them with respect to other pieces
of syntax). In these equations we make use of a meta-level
operation on processes $\mathsf{FN}(P)$ which returns the
set of free names (i.e. unrestricted names) in $P$.
For example $\mathsf{FN}(a.b.\mathbf{0}) = \{a, b\}$
but $\mathsf{FN}(a.\mathbf{0} \mid \nu b . (b . \mathbf{0})) =
\{a \}$.
%
\begin{align*}
  \begin{array}{rll}
  \nu a . P & \equiv P & \text{if} \; a,
  \overline{a} \not\in \mathsf{FN}(P)  \\
 \nu a . (P + Q) & \equiv \nu a . P + \nu a . Q \\
 \nu a. (P \mid Q) & \equiv \nu a. P \mid \nu a. Q & \text{if}\;(a
  \in \mathsf{FN}(P) \Rightarrow \overline{a} \notin \mathsf{FN}(Q)) 
 \wedge (a \in \mathsf{FN}(Q) \Rightarrow \overline{a} \not\in
  \mathsf{FN}(P))  \\
\nu a. (P \mid Q) & \equiv (\nu a. P) \mid Q &\text{if}\;a,
                                               \overline{a} \not\in
                                               \mathsf{FN}(Q)
\end{array}
\end{align*}
%
The first allows us to drop redundant restrictions. The second allows
us to distribute restriction inside choice. The third allows us to
distribute restriction inside parallel composition only if
$P$ and $Q$ cannot handshake via $a$. The fourth equation
states that we can shrink or grow the scope of a restriction
over processes that don't mention the restricted name.

We can apply structural congruences to simplify processes. For
example:
%
\begin{align*}
  \exampleLabel{}
  P + (\mathbf{0} \mid P) \equiv P
\end{align*}
%

\section{Bisimulation}

Often we want to be able to reason about whether two proceses behave
the same. For example, when programming, we might come up with a
refactoring or simplification which we want to ensure behaves in the
same way as the old program. Or, we might one to use one process
as a specification for another, giving a simple but inefficient
definition as the specification, against which we want to compare the
complicated but efficient implementation that we use. We can
describe process equivalence by comparing process behaviour:
that is, do the two proceses behave in the same way?

The notion of behaviour equivalence we study in this course is called
\emph{strong bisimulation}. This notion of behaviour equivalence
compares step-by-step what two proceses can do. Roughly, if we say
that a process $P$ is \emph{bisimilar} to a process $Q$ one process if
$P$ can make a reduction step with label $l$ then so can $Q$, and vice
versa, and the processes reached after reduction are also
bisimilar. This is encapsulated by the following definition,
which is actually a property of a relation.

\begin{definition}[Bisimulation]
  A relation $\mathcal{R}$ between processes is a bisimulation
  relation iff it satisfies the following: for all processes
  $P$ and $Q$ then:
  %
  \begin{align*}
    \begin{array}{rl}
P \mathcal{R} Q \qquad \Rightarrow & \forall P', l . \;\;\; P \xrightarrow{l} P' \;\; \Rightarrow \;\;
      (\exists Q'' . Q \xrightarrow{l} Q'' \; \wedge \; P' \mathcal{R}
      Q'') \\
      \wedge & \;\; \forall Q', l . \;\:\; Q \xrightarrow{l} Q'  \;\; \Rightarrow \;\; (\exists P'' . P \xrightarrow{l} P'' \; \wedge \; P'' \mathcal{R} Q')
      \end{array}
  \end{align*}
\end{definition}
%
\begin{definition}[Strong bisimulation]
  Two processes $P$ and $Q$ are \emph{bisimilar} if $P \sim Q$
  where $\sim$ is the largest bisimulation, defined:
  %
  \begin{equation*}
\sim = \bigcup \{\mathcal{R} \mid \mathcal{R} \; \textit{is a bisimulation} \}
  \end{equation*}
  (i.e. $\sim$ incorporates all possible bisimulation relations over processes).
\end{definition}
%
The following are some small examples:
%
\begin{itemize}
\item $(a.\mathbf{0} \mid \overline{a}.\mathbf{0})
\; \sim \; a.\overline{a}.\mathbf{0} + \overline{a}.a.\mathbf{0} +
\tau.0$;
\item $P \; \sim\;  a.Q$ where $P
{\mathrel{\mathop{=}\limits^{{def}}}} a.b.P$
and $Q {\mathrel{\mathop{=}\limits^{{def}}}} b.a.Q$;
\item $a.(b+c) \not\sim a.b + a.c$ (i.e., distributing
a prefix inside a choice leads to different behaviour).
\end{itemize}
%
The lecture gives more examples and motivates this definition further.

\end{document}